% 本模板由松鼠制作
% github地址:github.com/Sciurus365/PhysChemLab
% 2017年10月13日

\RequirePackage[l2tabu, orthodox]{nag}
\documentclass[UTF8]{article}


\usepackage{threeparttable}
\usepackage[zihao=-4]{ctex}
\usepackage{graphicx}
\usepackage{multicol}
\usepackage[a4paper]{geometry}
\usepackage{booktabs}
\usepackage{microtype}
\usepackage{siunitx}
\usepackage{ragged2e}
\usepackage{fontspec}
\usepackage{multirow}
\usepackage{mhchem}
\usepackage[square,sort,comma,numbers,super]{natbib}
\usepackage{fancyhdr}
\usepackage{url}





\graphicspath{{figures/}}
\geometry{left=2.0cm,right=2.0cm,top=2.0cm,bottom=2.0cm}
\setmainfont{Times New Roman}

\pagestyle{fancy}  
\lhead{物理化学实验报告}  
\chead{松鼠的制备纯化}  
\rhead{2017年10月13日}  
\lfoot{}  
\cfoot{\thepage}  
\rfoot{}  
\renewcommand{\headrulewidth}{0.4pt}  
\renewcommand{\footrulewidth}{0.4pt}  

% 温度 电动势 浓度和气压的快速输入
\newcommand{\swd}[1]{\SI{#1}{\degreeCelsius}}
\newcommand{\sdds}[1]{\SI{#1}{\volt}}
\newcommand{\snd}[1]{\SI{#1}{\mole \per \liter}}
\newcommand{\sqy}[1]{\SI{#1}{\kilo \pascal}}







\begin{document}
	
	\begin{titlepage}
		\vspace*{1cm}
		\begin{figure}[h]
			\centering
			\includegraphics[width=0.7\linewidth]{logo}
		\end{figure}
		
		\vspace*{0.5cm}
		
		\begin{center}
			\Huge{\textbf{物理化学实验报告}}
			
			\Large{松鼠的制备纯化}
		\end{center}
		
		\vspace*{0.5cm}
		
		\begin{table}[h]
			\centering	
			\begin{Large}
				\begin{tabular}{p{3cm} p{7cm}<{\centering}}
					姓\qquad 名: & 松鼠 \\
					\hline
					学\qquad 院: & 化学与分子工程学院 \\
					\hline
					学\qquad 号: & 1500011700 \\
					\hline
					分\qquad 组: & 第0组1号 \\
					\hline
					日\qquad 期: & 2017年10月13日 \\
					\hline
					室\qquad 温: & \swd{100}\\
					\hline
					气\qquad 压: & \sqy{100.0}\\
					\hline
					指导教师: & 吱吱吱\\
					\hline
				\end{tabular}
			\end{Large}
		\end{table}
	
	\vspace*{1cm}
	
	\textbf{摘要}\quad 本实验制备了松鼠并进行了纯化。
	
	\end{titlepage}

	
	
	
	\normalsize
	\section{实验操作}
	\subsection{实验药品及仪器}
	
	\subsection{实验步骤}
	
	\subsection{实验条件}
	实验条件数据见表\ref{tiaojian}。
	
	%三线表示例
	\begin{table}[htp]
		\centering
		\begin{threeparttable}
			\caption{实验室不同时间的气温和气压}\label{tiaojian}
			\begin{tabular} {lrr}
				\toprule
				时间 & 气温$T$/\si{\degreeCelsius} & 气压$p$/\si{\kilo\pascal}  \\
				\midrule
				1:00 & 100 & 100 \\
				2:00 & 100 & 100 \\
				3:00 & 100 & 100 \\
				\quad 平均值 & 100* & 100 \\
				\bottomrule
			\end{tabular}
		\small
		注:注释。
		\begin{tablenotes}
			\item[*] 熟了
		\end{tablenotes}
		\end{threeparttable}
	\end{table}
	
	\section{数据处理和结果}
	

	\section{思考与讨论}
	
	
	
	\section{实验结论}
	

	
	\bibliographystyle{ieeetr}
	\bibliography{mybib}
	


	
\end{document}